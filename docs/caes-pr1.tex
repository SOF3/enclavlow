\documentclass[a4paper, 12pt]{article}
\usepackage[top=0.2in, bottom=0.5in, left=0.2in, right=0.2in]{geometry}
\usepackage[utf8]{inputenc}
\usepackage[english]{babel}

\usepackage[bookmarksopen=true, hidelinks]{hyperref}
\usepackage{bookmark}
\bookmarksetup{numbered}
\hypersetup{colorlinks, linkcolor={black}}
\usepackage{cite}
\usepackage{listings}
\usepackage{xcolor}
\usepackage{fontspec}
\setmainfont{TeX Gyre Termes}
\usepackage{sourcecodepro}

\title{Data flow analysis for Uranus applications}
\author{Chan Kwan Yin (3035466978)}
\date{28 October 2020}

\lstdefinestyle{j}{
	keywordstyle=\color{magenta},
	basicstyle=\ttfamily\footnotesize,
	tabsize=4,
}

\begin{document}
\begin{titlepage}
	\begin{center}
		\vspace*{2em}
		\LARGE
		CAES9542 Progress Report 1

		\vspace*{1em}
		\Huge
		\textbf{Data flow analysis for Uranus applications}

		\Large
		\vspace{1.5em}
		\textbf{Chan Kwan Yin (3035466978)}

		\vspace{0.5em}
		28 October 2020
	\end{center}

	\vfill
	\paragraph{Abstract}
	Trusted Execution Environments (TEE) are useful for secure deployment on public clouds,
	protecting applications from privileged attacks,
	but defining enclave boundaries is not always a trivial task.
	It is easy to unintentionally leak sensitive data or derivations out of the enclave
	such that the host system is able to infer information about the sensitive data.
	This is especially difficult to detect in languages with complex control structures
	due to hidden control flow changes like exceptions.

	This project introduces Enclavlow, an information flow analysis tool
	for JVM-based projects using Intel SGX enclaves with the Uranus \footnote{
		Uranus: Simple, Efficient SGX Programming and its Applications.
		\url{https://doi.org/10.1145/3320269.3384763}
	} framework.
	It adopts a Dynamic Taint Analysis (DTA) approach
	and implements a set of security policies tailored for Uranus-based applications.
	The analysis tool is delivered as a Gradle plugin
	to be deployed as a continuous integration tool in Gradle-based projects.
\end{titlepage}

\thispagestyle{empty}
\tableofcontents
\listoffigures
\listoftables
\lstlistoflistings

\setcounter{page}0
\newpage
\setlength\parskip{1.5em}

\section{Motivation}

\section{Background}

\section{Objective}
\subsection{User interface}

\subsection{Algorithm}

\subsection{Security policies}

\section{Prior work}
\subsection{Glamdring}

\subsection{Phosphor}

\section{Methodology}
\subsection{Language}

\subsection{Flow analysis framework}

\subsection{}


\addcontentsline{toc}{section}{References}
\bibliographystyle{acm}
\bibliography{cite.bib}{}

\end{document}
