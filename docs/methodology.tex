\section{Methodology}\label{sec:methodology}
The main component of this project is the analysis framework,
which is conducted in the form of iterations
to fulfill security policies as specified in the integration tests.
Other parts such as Gradle plugin interfae,
although necessary for usage,
are not focus areas of this project, and hence will not be discussed further.

\subsection{Design}\label{subsec:design}
Since the adversary in the threat model
has arbitrary access to any untrusted memory and instruction,
the security policies of \pname{} differ slightly from typical information flow analysis.
For instance, the statement \code{a.b.c = d;} usually
does not propagate the effect of \code{d} to \code{a},
but since the adversary is capable of changing \code{b} to any memory location of its favour,
the security of this statement depends on whether \code{a} is trusted.

\pname{} adopts a flow graph approach,
where each node represents an element that may be leaked.

\begin{defin}
	Given a flow graph $(V, E)$, for all $x, y \in V$,
	$(x, y) \in E$ if and only if
	allocating $y$ outside an enclave reduces the indistinguishability of $x$
	even if all other nodes are removed.

	If the \q{even if} condition is removed, this is a transitive relation.
  The edges $\left\{(x, y), (x, z)\right\} \subset E$ imply that
  leaking \emph{either} $y$ \emph{or} $z$ already allows the adversary to distinguish $x$.
  However, there is no way to represent that $x$ is distinguishable if
  \emph{both} $y$ \emph{and} $z$ are distinguished.
  This limitation is resolved by creating additional nodes that delegate $x$.
\end{defin}

Note that this definition (\q{our definition}) differs slightly from
the usual definition of flow graphs (especially in \ac{DTA}),
where an edge $(x, y)$ represents the flow of information from $x$ to $y$~\cite{YinHeng2007Pcsi}.
In the usual definition, only read access for the adversary to $y$ is concerned,
while in our definition, the adversary possesses write access to $x$.

\pname{} constructs flow graphs in three stages
to reduce local opcodes to a cross-method flow graph.
Each method is first analyzed independently into a \ac{LFG}.
The \ac{LFG} is then contracted into a subgraph of only non-local nodes,
namely the \ac{CFG}.
The \ac{CFG} of each method is merged together to form the \ac{AFG},
on which enclave boundary intersections are identified to be leaks.

\subsubsection{\acf{CFG}}\label{subsubsec:cfg}
\pname{} constructs a \ac{CFG} for each method analyzed.
The \ac{CFG} contains the nodes listed in Table \ref{tab:cfg-nodes}.
In this article, the names for flow graph nodes are \fnname{underlined}.

\begin{table}
	\caption{Nodes in \ac{CFG}}
	\centering
	\begin{tabular}{|c|l|}
		\hline
		\fnname{Static} & Data located in static class fields \\ \hline
    \fnname{Param $x$} & Each parameter corresponds to a \fnname{Param} node \\ \hline
		\fnname{This} & The object on which a method was invoked \\ \hline
		\fnname{Return} & Information flow through the throw path \\ \hline
		\fnname{Throw} & Information flow through the return path \\ \hline
    \fnname{Source} & \code{sourceMarker} values \\ \hline
    \fnname{Sink} & \code{sinkMarker} values \\ \hline
    \fnname{Control} & See section \ref{subsubsec:control} \\ \hline
    Proxy nodes & \fnname{Param $y$}, \fnname{Return}, \fnname{Throw}
    and \fnname{Control} for each function call \\ \hline
	\end{tabular}
	\label{tab:cfg-nodes}
\end{table}

The \ac{CFG} is computed by contracting the \ac{LFG},
as described in the section \ref{subsubsec:lfg}.
After all methods in a class are evaluated,
the \ac{CFG}s of child methods called from the analyzed methods
are lazily evaluated as well.
All \ac{CFG}s are merged into an \ac{AFG},
joined using the function call nodes.

\begin{figure}
		\caption{Example \ac{CFG} for Listing~\ref{lst:SourceSinkExample}}
		\begin{center}
			\digraph[scale=0.5]{contract}{
				getSumParam0[label = "getSum\noexpand\n param \#0"];
				getSumControl[label = "getSum\noexpand\n control"];
				parseParam0[label = "parse\noexpand\n param \#0"];
				parseReturn[label = "parse\noexpand\n return"];
				parseControl[label = "parse\noexpand\n control"];
				computeSumParam0[label = "computeSum\noexpand\n param \#0"];
				computeSumReturn[label = "computeSum\noexpand\n return"];
				computeSumControl[label = "computeSum\noexpand\n control"];
				Source -> parseReturn [color = "red"];
				parseReturn -> computeSumParam0 [color = "red"];
				computeSumParam0 -> computeSumReturn [color = "red"];
				computeSumReturn -> Sink [color = "red"];
				getSumParam0 -> parseParam0;
				parseParam0 -> parseReturn;
				parseControl -> parseReturn;
				computeSumControl -> computeSumReturn;
				getSumControl -> parseControl;
				getSumControl -> computeSumControl;
			}
		\end{center}
    \label{fig:source-sink-contract}
\end{figure}

Figure~\ref{fig:source-sink-contract} shows an example of \ac{AFG}.
Unconnected nodes are omitted for brevity.

\subsubsection{\acf{LFG}\label{subsubsec:lfg}}
To construct the \ac{CFG}, a local flow graph is constructed
to identify the information flow between temporary variables.
Using \code{ForwardBranchedFlowAnalysis} from the Soot \cite{sootsurvivor} framework,
the analysis follows along the control flow of the program.
Java bytecode of the program to analyze is converted into Jimple code,
a \acf{3AC} language.
Each \ac{3AC} statement maps a \ac{LFG} to another \ac{LFG},
with the exception of \textit{branch} operations (\code{if}/\code{switch})
that map one flow graph to multiple branches
and \textit{merge} operations that map multiple branches to one.
For the case of loops,
the analysis repeats until the \ac{LFG} converges.

The \ac{LFG} extends the \ac{CFG} with the following additions:
\begin{itemize}
	\item Each local variable (some may only exist as intermediate values in source code)
		is allocated a node.
	\item Each branch has its own \fnname{Control} node.
\end{itemize}

For assignment statements,
rvalue nodes for the source, side effect nodes from the destination
and the current control flow contribute edges towards the lvalue nodes for the destination.
The rvalue and lvalue nodes depend on the expression type:

\paragraph{Literals}
As rvalues, these values do not leak any information.
They cannot appear as lvalues.

\paragraph{Local variables}
Local variables may be declared or temporary expressions.
As mentioned above, each local variable has its own node.

\paragraph{Parameters and \text{this}}
Jimple always assigns parameters to a local variable first,
so they can only appear as rvalues.
The only usage for parameters and \text{this} in Jimple code
is some statement like \code{r0 := @this: IagoAttack;},

\paragraph{Object field references}
A new \emph{projection} node is created for the field if it does not already exist.
An edge from the field to the object (\code{FIELD\_PROJECTION\_BACK\_FLOW}) is added.
For each assignment with the object value,
assignment edges are created from the equivalent fields to the new field.

For the case of static fields,
reads are always considered to have no leaks,
while writes are always considered to be leaks
as mentioned in the threat model section above.

\paragraph{Array references}
\code{a[b]} is treated as a special field of \code{a}.
If used as an lvalue or an rvalue, \code{b} also contributes as rvalue.

\paragraph{Array literal}
The array size contributes as an rvalue as it can be directly recovered from the array.
Since mutations have no effect on the size value provided in the array literal,
this is effectively the same as \code{array.size = inputSize;}.

\paragraph{Type casts, instanceof expressions and other operators}
As this only affects at the type level,
type casts and instanceof have the same semantics as their value operands.
Unary and binary operations have the same behaviour as
the union of rvalue nodes from their operands.

\paragraph{Method calls}
Calling methods creates blackbox proxy nodes to the nodes in their \ac{CFG},
with the exception of \code{sourceMarker} and \code{sinkMarker},
which are resolved as \fnname{Source} and \fnname{Sink} directly.
The \fnname{Sink} node is actually not necessary for analysis,
but is cosmetically helpful for interpreting sensitive data flow in the output.

For most expression types above, corresponding test cases can be found on the
GitHub repository under the directory \code{core/src/test}.

Since the mapping of flow graphs is deterministic,
overwriting a variable does not remove its body.

At exit points, the \ac{CFG} is constructed by considering the contract nodes
that flow into \fnname{This}, \fnname{Static}, \fnname{Sink},
all \fnname{Param}s and all blackbox proxy nodes.
For return and throw statements,
the flow into the returned/thrown value is also considered.

\subsubsection{Handling control flow}\label{subsubsec:control}
Sensitive data can be leaked through control flow,
such as in Listing \ref{lst:control-flow-leak}.
The code effectively copies the value of \code{secret} to \code{result}.
To handle this case,
a special node called \fnname{Control} is introduced
to denote the information that can be revealed through the control flow.

\IncludeCode{lst:control-flow-leak}{./ControlFlowLeak.java}
{Leak by control flow}{.3}3

The control node $\mathrm{CTRL}$ is defined with these semantics:

\begin{defin}
  For a local flow graph $(V, E)$, \begin{itemize}
    \item $(x, \mathrm{CTRL}) \in E$ implies that
      there exists a deterministic predicate $p$
      such that some code represented by $\mathrm{CTRL}$
      is executed if $p(x)$ is true.
    \item $(\mathrm{CTRL}_1, \mathrm{CTRL}_2) \in E$ implies that
      $\mathrm{CTRL}_2$ is a branch that executes only if $\mathrm{CTRL}_1$.
    \item $(\mathrm{CTRL}, y) \in E$ implies that
      leaking $y$ allows the adversary to distinguish whether $\mathrm{CTRL}$ was executed.
  \end{itemize}
\end{defin}

When branch statements are encountered,
a new \fnname{Control} node is created for each branch,
with an edge from the old node.
When the control flow converges,
the \ac{LCA} of the \fnname{Control} of each input \ac{LFG}
is selected as the new \fnname{Control}.

Each method call also involves its base control node;
for example, a \code{@JOCall} method always leaks everything including its control flow.
An empty \code{@JOCall} method leaks exactly just its own control,
so it is necessary to expose method control nodes as part of the contract graph.

\subsubsection{Aggregation}
The \ac{AFG} simply merges all \ac{CFG}s by linking proxy nodes together.
The presence of sensitive information leak is detected
by searching one of the following types of edges
in a \ac{DFS} from \fnname{Source}:
\begin{itemize}
  \item Return/throw paths of \code{@JECall}s
  \item Parameters/call context of \code{@JOCall}s
  \item Control flow of \code{@JOCall}s
  \item Assignment to \fnname{Static}
\end{itemize}


\subsection{Implementation details}\label{subsec:details}
Local and contract flow graphs are considered "dense" graphs,
and are represented using edge matrix.
Meanwhile, the aggregate flow graph has a much lower density of edges,
and an edge list is much more memory-efficient.

Since Soot uses \ac{LFG} equality as the predicate
to decide whether loops have to be analyzed again,
to avoid infinite loops, \fnname{Control} nodes and function calls
are excluded from consideration when comparing the graph.

\subsection{System Requirements}\label{subsec:system-requirements}
The logical code of this project is mostly implemented in Kotlin,
a \ac{JVM} language with more concise syntax than Java.
However, to ensure that the behaviour analyzed
is as explicit as possible,
all test cases are written in Java.

As Uranus was only tested against Linux systems,
this project does not intend to support other operating systems.
Furthermore, due to challenges with classpath detection,
only OpenJDK Versions 8 and 11 are supported currently.
Nevertheless, since \pname{} is just a developer tool,
its runtime is actually independent of that targeted by Uranus,
so it is possible to test support for other platforms in the future.

\pname{} is packaged as a Gradle plugin,
allowing developers to use it in projects with a Gradle toolchain.
However, the \code{enclavlow-core} subproject can be reused in other contexts,
such as Maven plugins, IDE plugins, etc.
With Kotlin toolchain integrated,
it is possible to include enclavlow code into Uranus for compile-time analysis.

\subsection{Flow analysis framework}\label{subsec:flow-analysis-framework}
Soot~\cite{sootsurvivor} was selected as the framework for conducting flow analysis.
Although multiple flow analysis systems using Soot already exist,
they are not designed against \ac{SGX} enclave protection.
\pname{} adopts more strict security policies
to prevent attacks from more privileged attackers,
unlike traditional information flow analysis
that mostly detects user input as the source of insecurity.

Soot accepts Java bytecode files (\code{*.class}) as input
and compiles them into Jimple \ac{3AC},
so inputs compiled from different \ac{JVM} languages are still compatible.
Nevertheless, some languages generate a lot of boilerplate code,
such as null assertion code from Kotlin,
which reduces readability of the report.

Other flow analysis frameworks were also considered,
such as Joana~\cite{joana} and JFlow~\cite{jflow}.
Soot was chosen due to its distinctively thorough documentation
and builtin support for call graph analysis.

\subsection{Testing}\label{subsec:testing}
This project uses JUnit 5 and \code{kotlin.test} framework
to conduct unit tests.
Test case classes are compiled together in the \code{:core:testClasses} task,
which declares dependency on the APIs \code{sourceMarker} and \code{sinkMarker},
but this is not necessary for actual usage.
