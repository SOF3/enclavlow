\begin{abstract}
	Trusted Execution Environments (TEE) protect applications from privileged attacks
	running on untrusted systems such as public clouds,
	but partitioning enclave boundaries is not always a trivial task.
	Partitions too small would leak data to the untrusted host system,
	while partitions too huge would result in unnecessarily large trusted computing base (TCB)
	that increases the risk of overflowing Enclave Page Cache (EPC).
	A passive analysis approach can be adopted where
	users annotate data as sensitive sources or sinks,
	and an analysis tool determines variables considered sensitive
	and compares it with the enclave boundaries declared.

	This project introduces \pname{}, an information flow analysis tool
	for JVM-based projects using Intel SGX enclaves with the Uranus
	\footnote{ Uranus: Simple, Efficient SGX Programming and its Applications.
	\url{https://doi.org/10.1145/3320269.3384763}
	} framework.
	It implements a set of security policies tailored for Uranus-based applications,
	and reports leaking variables or functions that could be run out of enclave.
	The analysis tool is delivered as a Gradle plugin
	to be deployed as a continuous integration tool in Gradle-based projects.
	The source code for \pname{} is released on \url{https://github.com/SOF3/enclavlow}.
\end{abstract}
