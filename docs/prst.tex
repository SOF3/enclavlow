\documentclass{beamer}
\usetheme{Frankfurt}
\usecolortheme{seahorse}
\setbeamercolor{background canvas}{bg=}

\usepackage[utf8]{inputenc}

\usepackage{sourcecodepro}
\usepackage[pdf]{graphviz}

\usepackage{tikz}
\usepackage{xcolor}
\usepackage{soul}
\definecolor{code}{rgb}{0.95, 0.95, 0.95}
\sethlcolor{code}
\newcommand{\code}[1]{\colorbox{code}{\texttt{\footnotesize #1}}}

\usepackage{csquotes}
\newcommand{\q}[1]{\enquote{#1}}

\usepackage{listings}
\usepackage{ifthen}
\tikzstyle{highlighter} = [
  lime,
  line width = \baselineskip-0.3em,
]
\newcounter{highlight}[page]
\newcommand{\tikzhighlightanchor}[1]{\ensuremath{\vcenter{\hbox{\tikz[remember picture, overlay]{\coordinate (#1 highlight \arabic{highlight});}}}}}
\newcommand{\bh}[0]{\stepcounter{highlight}\tikzhighlightanchor{begin}}
\newcommand{\eh}[0]{\tikzhighlightanchor{end}}
\AtBeginShipout{\AtBeginShipoutUpperLeft{\ifthenelse{\value{highlight} > 0}{\tikz[remember picture, overlay]{\foreach \stroke in {1,...,\arabic{highlight}} \draw[highlighter] (begin highlight \stroke) -- (end highlight \stroke);}}{}}}
\lstdefinestyle{j}{
  keywordstyle=\textbf,
  basicstyle=\ttfamily\scriptsize,
  tabsize=4,
  numbers=left,
  escapechar=`,
}
\newcommand{\IncludeCode}[2]{
  \lstinputlisting[style=j, language=java, caption={#2}]{#1}
}

\usepackage{cite}

\AtBeginSection[]{
  \begin{frame}
    \begin{beamercolorbox}[sep=0.5em, center, shadow=true]{title}
      \usebeamerfont{title}\insertsectionhead\par
    \end{beamercolorbox}
  \end{frame}
}

\title{Data flow analysis for Uranus applications}
\author{Chan Kwan Yin}

\begin{document}

\frame{\titlepage}

\begin{frame}
  \frametitle{Outline}
  \tableofcontents
\end{frame}

\section{Background}

\begin{frame}
  \frametitle{SGX Enclaves}
  \begin{columns}
    \begin{column}{0.5\textwidth}
      \begin{itemize}
        \item Servers outsourced to third-party cloud providers
        \item Threat model: Adversaries with privileged access to OS, BIOS or hardware
        \item Enclave protects both code and memory from these adversaries
      \end{itemize}
    \end{column}
    \begin{column}{0.5\textwidth}
      \digraph[scale=0.35]{EnclaveExample}{
        compound = true;
        %
        subgraph cluster_enclave {
          label = "enclave";
          color = "red";
          %
          secret_key [label = "secret key"];
          decrypted [label = "decrypted input"];
          decryption [shape = "box";]
          anonymization [shape = "box"];
          secret_key -> decryption;
          decryption -> decrypted;
          decrypted -> anonymization;
        }
        %
        encrypted_input [label = "encrypted input"];
        output [label = "output"];
        encrypted_input -> decryption;
        anonymization -> output;
        %
        adversary [label = "Adversary", shape="diamond"];
        adversary -> decrypted [lhead = cluster_enclave, color = "red", dir = "none"];
        adversary -> output [dir = "none", color = "forestgreen"];
        encrypted_input -> adversary [dir = "none", color = "forestgreen"];
      }
    \end{column}
  \end{columns}
\end{frame}

\begin{frame}
  \frametitle{Uranus \cite{uranus}}
  \begin{itemize}
    \item OpenJDK fork that supports Intel SGX
    \item Methods marked as \code{@JECall} enters enclaves until return
    \item Methods marked as \code{@JOCall} exits enclaves until return
    \item Useful for integration with libraries like Hadoop and Spark
    \item Question: Where should \code{@JECall} and \code{@JOCall} be placed?
    \item Question: Is code in these libraries safe as enclave code?
  \end{itemize}
\end{frame}

\begin{frame}[t]
  \frametitle{The problem: Performanc/Security Tradeoff}
  \begin{itemize}
    \item More code outside enclave:
      \begin{itemize}
        \item Increased risk of leaking protected data
        \item Some leaks may come from unexpected side channels
      \end{itemize}
    \item More code into enclave:
      \begin{itemize}
        \item Limited EPC (Enclave Page Cache)
          \begin{itemize}
            \item Up to 100 MB of EPC
            \item Out of memory $\implies$ extremely slow swap
            \item JVM applications especially memory-greedy
          \end{itemize}
        \item Principle of Least Privilege
          \begin{itemize}
            \item Vulnerabilities in enclave code bypass enclave protection
            \item Vulnerabilities in code outside must only use specific entry points
            \item Reduce attack surface
          \end{itemize}
      \end{itemize}
  \end{itemize}
\end{frame}

\begin{frame}[fragile]
  \frametitle{The solution}
  \begin{itemize}
    \item \textit{enclavlow} \footnote{coined from the words "enclave" and "flow"}:
      an information flow analysis tool
    \item Sources of sensitive data marked with \code{sourceMarker}
    \item Anonymization marked with \code{sinkMarker}
    \item Identity functions; expected to be optimized them away by JIT
  \end{itemize}

  \begin{lstlisting}[style=j, language=java]
  @JECall
  static int process(byte[] encrypted) {
    byte sum = 0;
    byte[] password = `\bh`sourceMarker`\eh`(new byte[]{1, 2, 3, 4, 5, 6});
    byte[] decrypted = decrypt(password, encrypted);
    for(byte b : decrypted) sum ^= b;
    return `\bh`sinkMarker`\eh`(sum);
  }
  \end{lstlisting}
\end{frame}

\section{Approach}
\begin{frame}
  \frametitle{Intuition: Trivial ways of leaking data}
  \begin{itemize}
    \item Returning/throwing out of a \code{@JECall}
    \item Passing into a \code{@JOCall}
    \item Assigning to a static field
  \end{itemize}
\end{frame}

\begin{frame}
  \frametitle{Intuition: Non-trivial ways of leaking data}
  \begin{itemize}
    \item Assigning to an outside-enclave object:\\
      \code{@JECall void x(Manager m) \{} \\
      \code{~~m.value = sourceMarker(secret); \} } \\
    \item Leaking control flow into variables:\\
      \code{for(int i=0; i < sourceMarker(secret); i++) outside++;}
    \item Implicit exceptions:\\
      \code{int[3] array; array[sourceMarker(secret)] = 1;}
  \end{itemize}
\end{frame}

\begin{frame}
  \frametitle{Flow analysis}
  \begin{itemize}
    \item Analysis framework: Soot \cite{sootsurvivor}
    \item Each method is analyzed independently
      \begin{itemize}
        \item Function calls treated as blackboxes
        \item 
      \end{itemize}
    \item 
  \end{itemize}
\end{frame}

\begin{frame}
  \frametitle{Flow graph}
  \begin{columns}
    \begin{column}{0.5\textwidth}
      (missing graphic)
    \end{column}
    \begin{column}{0.5\textwidth}
      (missing graphic)
    \end{column}
  \end{columns}
\end{frame}

\section{Limitations \& Future Work}
\begin{frame}
  \frametitle{Detecting implicit exceptions}
  \begin{itemize}
    \item Conditional runtime errors leaks data
    \item Extensively researched in both academia and industry
    \item Solutions usually achieved at the language level,
      e.g. \code{@NonNull}, \code{@Size}
    \item Good practices: all exceptions should be caught at enclave boundary anyway!
    \item Similar: \code{x + secret - secret}
  \end{itemize}
\end{frame}

\begin{frame}
  \frametitle{Tackling polymorphism}
  \begin{itemize}
    \item Computing all combinations of instance classes takes exponential time.
    \item \textit{\q{Java workloads don't fit into enclave programming paradigms}} \cite{civet}
    \item Detecting possible paths subclasses reduces complexity,
      but still not perfect.
  \end{itemize}
\end{frame}

\begin{frame}
  \frametitle{Integration into Uranus}
  \begin{itemize}
    \item Uranus disallows reads/writes of objects outside enclave
      without \code{SafeGetField} etc
    \item Runtime overhead of checking object location
    \item \textit{enclavlow} to perform this analysis at compile-time
  \end{itemize}
\end{frame}

\section{Conclusion}
\begin{frame}
  \begin{itemize}
    \item Assist with decisions on performance/security tradeoff
    \item Incorporated into Uranus
    \item Applications in big data industry
    \item Room for improvement on specialization cases
  \end{itemize}
\end{frame}

\section{Appendix}
\begin{frame}
  \frametitle{References}
  \scriptsize
  \bibliographystyle{acm}
  \bibliography{cite}
\end{frame}

\end{document}
