\section{Results and Discussion}
\subsection{Difficulties and Limitations}
The principles of OOP imply that a method called may be swapped with another subclass
that performs different actions than the current one.
Although Uranus prevents the adversary
from passing arbitrary malicious code into the enclave memory,
it is still possible to pass obejcts of unexpected subclass
through the \code{@JEcall} boundary.
Consider listing \ref{lst:oop-subst-attack} for example.
If \code{cs} is passed with a \code{substr} implementation
that writes its parameters to a static variable,
the function would leak the length of the security-sensitive secret,
which is not desirable.
To correctly solve the vulnerability of OOP substitution,
it is necessary to perform call graph analysis on the actual classes passed to the method,
which involves more complex framework level work.

\IncludeCode{lst:oop-subst-attack}{./OopSubstAttack.java}
{Example attack through OOP substitution}

Despite optimizations and simplifications,
it is still not possible to perform 100\% accurate information flow analysis
within efficient time complexity \cite{SmithGeoffrey2007PoSI}.
For example, this project merges conditional branches together
by taking the union of flow graphs,
resulting in easy false positive rates.
Listing \ref{lst:known-false-positives} enumerates a number of false positives
incorrectly identified by \pname{},
which are not going to be fixed because of the unlikeness of use.

\IncludeCode{lst:known-false-positives}{./KnownFalsePositives.java}
{Known false positives}

It is also impossible to analyze further than the JNI level,
since analyzing across JNI boundary implies
the need to interact with a native analysis tool,
which is entirely out of scope of this project.
Since Uranus effectively denies system calls,
the contract is always assumed to be $\left\{ (p, \text{return}) : p \in \text{params} \right\}$.

For the sake of consistency with Uranus,
it was originally intended to expose \code{sourceMarker} and \code{sinkMarker}
as annotations on local variables instead of method calls.
However, JLS 9.6.1.2 explicitly stated that
\q{an annotation on a local variable declaration is never retained in the binary representation}
\cite{jls},

At the technical aspect,
multiple technical challenges were encountered when using Soot.
Since Soot was designed to be a command line tool,
it does not support direct calling from other environments very well.
In particular, the entrypoint of Soot API is the \code{soot.Main.main()} method,
which is the standard CLI interface in Java.
As a result, extra time is spent on clearing global states used by Soot.
Furthermore, due to restrictions in the Soot framework,
each tested Java method must be run in a separate JVM,
resulting in poor testing performance.
