\section{Discussion}\label{sec:discussion}
There are various unimplemented features in this project,
which can be tackled in future research.

\subsection{Limitations}\label{subsec:difficulties-and-limitations}
As is every code analysis system,
perfect detection is almost impossible.
This subsection will show that
the requirements of \pname{} is a \emph{superset} of the common analysis tools
in terms of problems to identify.

\subsubsection{Polymorphism}\label{subsubsec:polymorphism}
The principles of \ac{OOP} imply that the receiver of a method call
may be swapped with a compatible implementation in another subclass
that performs different actions than the current one.
Although Uranus prevents the adversary
from passing arbitrary malicious code into the enclave memory,
it is still possible to pass objects of unexpected but trusted subclass
through the \code{@JECall} boundary.
Alternatively, an alternative form of Iago attack
passes originally impossible combination of subtypes to the function,
which could also introduce attack vectors.
Consider Listing~\ref{lst:oop-subst-attack} for example.
If \code{cs} is passed with a \code{substr} implementation
that writes its parameters to a static variable,
the function would leak the length of the security-sensitive \code{secret},
which is not desirable.
To correctly solve the vulnerability of \ac{OOP} substitution,
it is necessary to perform call graph analysis on the actual classes passed to the method,
which involves more complex framework level work.

\IncludeCode{lst:oop-subst-attack}{./OopSubstAttack.java}
{Iago attack through substitution}{.6}{2.5}

Due to time constraints,
polymorphism handling is not implemented in this project.
Instead, a \emph{degenerate} contract graph
with the edge set $\left\{(\text{\fnname{Param}}~x, \text{\fnname{Return}}) : x\right\}$
is used as placeholder for abstract methods,
and possible subclass overrides are not yet considered.

\subsubsection{Native methods}\label{subsubsec:jni}
Similar to abstract methods,
\ac{JNI} methods are also not easy to analyze.
The implementation of \ac{JNI} methods is only available in the form of native dynamic libraries,
and reverse engineering such libraries is unnecessary, tedious and sometimes illegal.
A much simpler but accurate approach
is to analyze the native methods in the original languages they were written in,
integrating with tools such as Glamdring \cite{glamdring};
for the native methods within Java standard library,
it is possible to precompute the contract of each of them manually
and ship these with the software.
Similar to above, this task is omitted due to time constraints;
degenerate contracts are used as placeholder instead.
This assumption is inaccurate;
a common counterexample is the \code{System.arraycopy} method,
which leaks the values of the integer parameters into the destination array parameter
since the amount of elements changed leaks the integers involved.

\subsubsection{Duplicate code}\label{subsubsec:duplicate-code}
Despite optimizations and simplifications,
it is still not possible to perform perfectly accurate information flow analysis
within efficient time complexity~\cite{SmithGeoffrey2007PoSI}.
For example, this project merges conditional branches together
by taking the union of flow graphs,
resulting in easy false positive cases.
Listing~\ref{lst:known-false-positives} is an example of this:
the return statement appears in both arms,
so according to the mechanism explained above, the control flow shall leak
the branching condition \code{secret} to the return path.
Nevertheless, but the returned value is in fact always \code{x}.
Since this is a minor use case, this bug is considered not worth fixing.
In fact, duplicated code in multiple branch arms
is often regarded as an antipattern as well. % citation needed

\IncludeCode{lst:known-false-positives}{./KnownFalsePositives.java}
{False positive by duplicate code}{.6}{3}

Similarly, self-anonymized sensitive data are not identified correctly,
such as in Listing~\ref{lst:self-anonymized}.
It is believed that such errors are easy for a human to discover
and could be marked \code{sinkMarker} directly,
or simply copying \code{a} to another variable,
so these false positives do not have major effect on usability.
Note also that mutably sensitive data may be an antipattern
as the user may accidentally leak it in the future
by moving a line of code incorrectly.

\IncludeCode{lst:self-anonymized}{./SelfAnonymized.java}
{False positive by self-anonymization}{.6}{3}

\subsubsection{Implicit exceptions}
Apart from explicit \code{throw} statements,
runtime exceptions can also be raised when invalid operations are performed,
leaking information through control flow.
Consider Listing \ref{implicit-exceiption}.
While the mechanism of \pname{} does not discover any bugs,
the method throws an ArrayIndexOutOfBoundsException
if and only if \code{a >= index},
so an adversary could pass varying values for \code{a}
to binary-search the value of \code{index}.

\IncludeCode{implicit-exceiption}{./ImplicitException.java}
{Implicit exception}{.4}{3}

This is difficult to notice even for an alert developer.
Solutions are unfortunately usually achieved at the source code level,
such as using annotations like \code{@NotNull} and \code{@Size}.
As this project does not intend to reinvent the wheel,
prevention of such vulnerabilities is left as the task for other source-level tools.
In fact, such exceptions tend to cause other security issues in addition to enclave security,
well known as the "billion dollar problem" \cite{nullbillion}.
It is likely more efficient to try to avoid such unexpected behaviour
by fusing the source code before it gets compiled.

\subsubsection{Readability}
For the sake of consistency with Uranus,
it was originally intended to expose \code{sourceMarker} and \code{sinkMarker}
as annotations on local variables instead of method calls.
However, \ac{JLS} 9.6.1.2 explicitly stated that
\q{an annotation on a local variable declaration
is never retained in the binary representation}~\cite{jls},
so a method call based approach is used instead.
It is expected that \ac{JIT} optimization removes the cost
involved from an extra method call at \ac{JIT} compile time.

This also leads to readability issues.
The current HTML output is unable to reveal more information than the \ac{CFG}
because of the lack of information about the source code,
like local variable names and line numbers.
In the \ac{LFG}, generated variable names like \code{r0}
and control marker names like \code{control13} are used instead,
but they are not intuitive for the reader
and become hard to interpret when the method grows large.
The only solution for this problem is to provide additions at the source code level,
but since \pname{} takes inputs in the form of Java class binaries,
this is considered out of scope of the project.

\subsection{Recommended future research}\label{subsec:recommended-future-research}
There are multiple areas in which this project can be extended.

\pname{} applies handwritten heuristics to identify leaks.
Some special edges, such as field projection, are not very well-defined.
This reduces the reliability of \pname{} in terms of robustness in targeted attacks,
which is an important feature for security analysis.
A formal proof through tools like Coq~\cite{coq}
can be utilized to ensure that the \ac{LFG} construction does not miss marginal cases.

The project can also be used to improve Uranus performance.
Currently, to ensure enclave confidentiality,
Uranus requires enclave code accessing untrusted memory to
use Uranus's untrusted-memory \ac{API}
like \code{SafeGetField} and \code{SafeWriteField}\cite{uranus},
which checks the memory address against enclave bounds at runtime.
The static analysis in \pname{} allows Uranus to validate these bounds at compile time,
hence avoiding the runtime bounds-checking cost and improve performance.
Note that JIT optimization is not able to detect unnecessary bounds checking.
Since \code{enclavlow-core} exposes the \ac{AFG} result fully,
it shall be possible for the compiler in Uranus
to directly invoke this library to analyze the program being compiled.
