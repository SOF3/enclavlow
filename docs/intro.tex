\section{Introduction}\label{sec:introduction}

\subsection{Background}\label{subsec:background}
With the rise of third-party public cloud servics
such as AWS ~\cite{aws} and Microsoft Azure ~\cite{azure},
there is increasing demand for trusted execution where
applications are protected from
attackers with privileged access to the hardware or software.
Modern hardware offer TEE technologies,
such as SGX in Intel CPUs,
with which trusted execution code and sensitive data
are processed in secure \q{enclaves},
which is protected at hardware level to prevent access
from other hardware or software layers.

One significant application of TEEs is in big data processing,
where confidential user data are processed,
and protection from cloud providers may be necessary
for complianc with privacy regulations such as GDPR ~\cite{gdpr}.
However, a significant subset of such applications are written
using languages that use JVM as the runtime,
such as Hadoop ~\cite{apachehadoop} and Spark ~\cite{apachespark}
Recently, Uranus, a system for
writing SGX applications in Java languages, was released ~\cite{uranus}.
It provides simple interface for SGX,
where users annotate methods with \code{@JECall} and \code{@JOCall}
to move control flow into or out of enclaves.
It is the responsibility of the user to determine the correct positions
for the \code{@JECall} and \code{@JOCall} annotations,
namely the enclave boundary partitioning.
Since JVM, compared to native applications running on the CPU,
involves an entirely different approach
with regard to software development and distribution,
the tools applicable for native applications are mostly incompatible with JVM,
introducing the corresponding new research areas.

Running the whole application within an SGX enclave is undesirable for two reasons.
First, this violates the principle of least privilege,
where the whole application becomes possible attack surface
for adversaries to compromise protected data ~\cite{glamdring}.
Second, this implies all memory used by the application
are placed in the enclave memory (the EPC),
which is restricted to 100 MB before significant performance degrading
(\q{1,000X slowdown compared to regular OS paging}) ~\cite{uranus}.
On the other hand, if the enclave is smaller than necessary,
adversaries can either obtain sensitive data directly or
infer sensitive characteristics of them indirectly.

This project presents \pname{}
\footnote{\q{enclavlow} is a new term coined from the words \q{enclave} and \q{flow}.},
an information flow analysis tool
for identifying data leak from enclaves.
The user first wraps sensitive data sources in \code{sourceMarker} and sinks in \code{sinkMarker}.
The tool performs information flow analysis from \code{sourceMarker} variables,
identifying the ways that data from such variables are leaked
to the system outside the executing enclave
without first passing through a \code{sinkMarker} variable.
The tool compiles a report in HTML format that summarizes the following:
\begin{itemize}
	\item \textbf{Data leak}:
		The report displays the lines of code on which sensitive data are moved
		into areas accessible by privileged adversaries.
		It demonstrates the path from the \code{sourceMarker} expression to the point of leak.
	\item \textbf{Redundant protection}:
		The report lists the functions that could not hold any sensitive data in its local variables,
		hence should be moved out of the enclave partition.
\end{itemize}

\pname{} is shipped as a Gradle plugin,
providing a Gradle task that
takes the \code{*.class} files compiled in the \code{classes} task
and generates the report for the analysis from those classes.

\subsection{Prior art}\label{subsec:prior-art}
Information flow analysis is not a new technology in the field.
While this project analyzes JVM code using SGX enclaves,
prior research on \emph{native} code \emph{automatic partition} was found.

Glamdring ~\cite{glamdring} is a C framework that
automatically selects the minimal SGX enclave boundaries
based on user requirements specified through C pragma directives.
However, since the process is fully automated,
it has a lower tolerance of false positives,
which increases the risk of unintentional data leak.
This project, unlike Glamdring, will only perform analysis but not automatic partitioning,
allowing for greater false positive tolerance.

Phosphor ~\cite{BellJonathan2014Pidd} is a DTA framework
that modifies Java bytecode to add tags to sensitive data at runtime
and check if such tags are leaked.
Although dynamic taint is more accurate,
this project prefers a static analysis approach,
which enables developers to identify sensitive regions at compile time
without the need to feed concrete data into methods.

Civet ~\cite{civet} % TODO
